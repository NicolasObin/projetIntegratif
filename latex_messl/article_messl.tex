\documentclass[a4paper,12pt,twocols]{article}
\usepackage[T1]{fontenc}
\usepackage[frenchb]{babel}
\usepackage[utf8]{inputenc}
\usepackage{lmodern}
\usepackage[babel]{csquotes}
\MakeAutoQuote{«}{»}
\pagestyle{headings}

\title{Bout sur MESSL}
\author{jmatthieu}

\begin{document}
\maketitle

\section{introduction}
MESSL incorpore une technique de séparation de sources basée sur un masquage temps-fréquence, en regroupant les points temps-fréquence selon les indices inter-auraux de différence de phase et de niveau. L'hypothèse est faite qu'un point temps-fréquence n'est dominé que par une seule source, cette supposition est généralement valable pour les mixtures de signaux de parole [cite1] Le système décrit chaque source par un modèle statistique des indices inter-auraux. C'est lors de cette modélisation que l'algorithme EM (Expectation-Maximization) [cite2] est utilisé.
\section{modèle}
Nous présupposons l'existence de $i$ sources, et nous discrétisons les décalages de phases possibles, notés $\tau$. Les indices ILD et IPD sont modélisés par des gaussiennes, et sont supposés conditionnellements indépendants.
$$ formule des modeles,eq5a8$$
Faire un bout sur l'algo EM/ mise en place et equations.
Puis cela permet d'extraire le masque
$$equation26$$
Celui ci peut être amélioré en ajoutant des présupposés sur les sources mais cela n'a pas été abordé.

\section{biblio}
cite1 \\ O. Yilmaz and S. Rickard, “Blind separation of speech mixtures via
time-frequency masking,” IEEE Trans. Signal Process., vol. 52, no. 7,
pp. 1830–1847, July 2004.


cite2 \\
Gaussian mixture models and the EM algorithm, Ramesh Sridharan\\
The EM algorithm : a short tutorial, Sean Borman\\
CS229 lecture notes, Stanford University, Andrew Ng\\
\end{document}
